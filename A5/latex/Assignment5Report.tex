\documentclass{article}
\usepackage{amsmath,graphicx,amssymb,amsthm,url}

\title{Implementation Assignment 5}
\date{\today}
\author{Rong Yu and Finn Womack}

\begin{document}
	\maketitle
	\section{Implementing the iterative value algorithm}

	First, we loaded the data into matlab using the dlmread function which returned a matrix of all the values appending zeros where needed. (I.e. The first row was m and n followed by several zeros, followed by the next several rows containing the action matrices smashed together, and the last row containing the reward vector.) We then reformatted the data into 4 variables and ran it through the iterative value function.
	
	\section{Results}
	\subsection{$\beta = 0.1$}
	
	Using a discount factor of $\beta = 0.1$ we found the following optimal utility vector:
	
	\begin{equation}
	U
	=
	\begin{bmatrix}
    0.1001\\
	0.0090\\
	0.0088\\
	0.0090\\
	1.0010\\
	0.0068\\
	0.0683\\
	0.0086\\
	0.0100\\
	0.0896\\
	\end{bmatrix}
	\end{equation}
	
	as well as an optimal policy of:
	
	\begin{equation}
	V
	=
	\begin{bmatrix}
	4\\
	4\\
	3\\
	1\\
	1\\
	1\\
	2\\
	3\\
	2\\
	4\\
	\end{bmatrix}
	\end{equation}
	
	where the ith entry of each vector corresponds to the optimal policy/utility of the ith state.
	
	\subsection{$\beta = 0.9$}
	
	Next, using a discount factor of $\beta = 0.9$ we found the following optimal utility vector:
	
	\begin{equation}
	U
	=
	\begin{bmatrix}
	3.3210\\
	2.9234\\
	2.8914\\
	2.9234\\
	3.6900\\
	2.8407\\
	3.1564\\
	2.9071\\
	2.9889\\
	3.2482\\
	\end{bmatrix}
	\end{equation}
	
	as well as an optimal policy of:
	
	\begin{equation}
	V
	=
	\begin{bmatrix}
	4\\
	4\\
	3\\
	1\\
	1\\
	1\\
	2\\
	3\\
	2\\
	4\\
	\end{bmatrix}
	\end{equation}
	
	where the ith entry of each vector corresponds to the optimal policy/utility of the ith state. Something interesting to note is that while the discount factors gave different optimal utility vectors they both returned the same optimal policy.
	
	%\bibliography{myCitations} 
	%\bibliographystyle{abbrv}
	
\end{document} 